\documentclass[a4paper]{article}

\usepackage{amssymb, amsmath, amsfonts, amsthm}
\usepackage{cite} % for citing from *.bib
\usepackage{color}
\usepackage{graphicx}
\usepackage{subfigure}
\usepackage{wrapfig}
\usepackage{xcolor}

\usepackage{tikz}

\usetikzlibrary{arrows,matrix,fit}

%% For theorems and the like: %%
\theoremstyle{plain}
\newtheorem{thm}[equation]{Theorem}
\newtheorem{Lemma}[equation]{Lemma}
\newtheorem{Proposition}[equation]{Proposition}
\newtheorem{Corollary}[equation]{Corollary}

\theoremstyle{definition}
\newtheorem{defi}[equation]{Definition}
\newtheorem{Remark}[equation]{Remark}
\newtheorem{Example}[equation]{Example}

%% Document %%
\begin{document}

\begin{defi}
	We say that two sequences of projection matrices \( (P^1, \ldots, P^k) \) and
	\( (\hat{P}^1, \ldots, \hat{P}^k) \) are \emph{projectively equivalent} if there
	exists an invertible matrix \( H \) and scalars \( \lambda_i \) such that
	\[
		\hat{P}^i = \lambda_i \cdot P^i H
	\]
	for all \( i \).
\end{defi}

\begin{defi}
	Let \( P^i \) and \( \hat{P}^i \) be matrices of size \( (m_i + 1) \times (n+1) \).
	Also let the two matrices \( P \) and \( \hat{P} \) be made of the \( k \) blocks \( P^i \)
	and \( \hat{P}^i \) respectively. Then we say \( P \) and \( \hat{P} \) are
	\emph{block projectively equivalent} if there exists an invertible matrix \( H \) of size
	\( (n+1) \times (n+1) \) and scalar matrices \( \lambda_i I_i \) of size
	\( (m_i + 1) \times (m_i + 1) \) so that
	\[
		\hat{P} = \text{diag}(\lambda_1 I_1, \ldots, \lambda_k I_k) P H.
	\]
\end{defi}

\begin{thm}
	Let \( P \) be a matrix with \( k \) blocks \( P^i \) of size \( (m_i+1) \times (n+1) \).
	Also let \( (\alpha_1, \ldots, \alpha_k) \) be an ordered partition of \( n+1 \).
	If at least one \( m_i \) is greater than one, then the matrix \( P \) is determined
	up to block projective equivalence by the collection of its minors, chosen with
	\( \alpha_i \) rows from each \( P^i \).
\end{thm}

\end{document}